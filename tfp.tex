\documentclass[sigplan,balance=true,pbalance=true,natbib=false]{acmart}
%% ∞
%% ⊂
%% …
%% ₀
%% ₁
%% ₂
%% ₃
%% ₁₂₃₄₅₆
\usepackage{fontspec,newunicodechar}
\usepackage{hyphenat} %% electromagnetic\hyp{}endioscopy
\usepackage[shortcuts]{extdash}
\usepackage{microtype} %% provides tighter text formatting
\usepackage{flushend}
\usepackage{balance}
\usepackage{polyglossia} %% for xelatex use polyglossia, needed for csquotes
\setdefaultlanguage[variant=american]{english}
\usepackage{fancybox}

\PassOptionsToPackage{cache=false,newfloat=true}{minted}
\PassOptionsToPackage{unicode}{hyperref}

\usepackage[capitalize,nameinlink]{cleveref} %% adds \crefrange{}{} and \cpagerefrange{}
%% \BibTeX command to typeset BibTeX logo in the docs
\AtBeginDocument{%
  \providecommand\BibTeX{{%
    Bib\TeX}}}

\RequirePackage[
  backend=biber,
  natbib=true,
  datamodel=acmdatamodel,
%%  style=acmnumeric,
  ]{biblatex}

%% Declare bibliography sources (one \addbibresource command per source)
\addbibresource{tfp.bib}

\overfullrule=0pt

\usepackage{nag} %% should complain about old and outdated commands
\usepackage{minted}
\usemintedstyle{bw}

\usepackage[strict,autostyle]{csquotes} %% enquote &c that doesn't interfere w/cdlatex usage
%% http://ftp.lyx.org/pub/tex-archive/macros/latex/contrib/csquotes/csquotes.pdf
%% \usepackage[xspace]{ellipsis} they suggest using xetex instead
%% \usepackage{setspace} %% set the spacing (single double, etc)

\setmonofont{DejaVuSansMono}[Scale=MatchLowercase]
\tracinglostchars=2
\usepackage[defaultlines=3,all]{nowidow} %% Eliminates widow and orphans
\setlength {\marginparwidth }{2cm}
\usepackage{todonotes}

\newmintinline[rackinline]{racket}{}
\newmintinline{prolog}{}

\setcopyright{acmcopyright}
\copyrightyear{2023}
\acmYear{2023}
\acmDOI{XXXXXXX.XXXXXXX}

\acmConference[TFP '23]{Symposium on Trends in Functional
  Programming}{January 13-15, 2023}{Boston, Massachusetts}

\acmPrice{15.00}
\acmISBN{978-1-4503-XXXX-X/18/06}

\begin{document}

\title{Nearly Macro-free microKanren}

\author{Jason Hemann}
\email{jason.hemann@shu.edu}
\affiliation{%
  \institution{Seton Hall University}
  \country{USA}
}
\orcid{0000-0002-5405-2936}
\author{Daniel P. Friedman}
\email{dfried@indiana.edu}
\affiliation{%
  \institution{Indiana University}
  \country{USA}
}
\orcid{0000-0001-9992-1675}

\renewcommand{\shortauthors}{Hemann and Friedman}

\begin{abstract}

  We describe changes to the microKanren implementation that make it
  more practical to use in a host language without macros. With
  some modest runtime features common to most languages, we
  show how an implementer lacking macros can come closer to the
  expressive power that macros usually provide---with varying degrees
  of success. The result is a still functional microKanren that
  invites slightly shorter programs, and is relevant even to
  implementers that enjoy macro support. For those without it, we
  address some pragmatic concerns that necessarily occur without
  macros so they can better weigh their options.

\end{abstract}

\begin{CCSXML}
<ccs2012>
   <concept>
       <concept_id>10011007.10011006.10011008.10011009.10011015</concept_id>
       <concept_desc>Software and its engineering~Constraint and logic languages</concept_desc>
       <concept_significance>500</concept_significance>
   </concept>
   <concept>
       <concept_id>10003752.10003790.10003795</concept_id>
       <concept_desc>Theory of computation~Constraint and logic programming</concept_desc>
       <concept_significance>300</concept_significance>
   </concept>
 </ccs2012>
\end{CCSXML}

\ccsdesc[500]{Software and its engineering~Constraint and logic languages}

%%
 %% Keywords. The author(s) should pick words that accurately describe
%% the work being presented. Separate the keywords with commas.
\keywords{logic programming, miniKanren, DSLs, embedding, macros}

\maketitle

\section{Introduction}

Initially we designed microKanren~\cite{hemann2013muKanren} as a
compact relational language kernel to undergird a miniKanren
implementation. Macros implement the surrounding higher-level
miniKanren operators and surface syntax.\@ microKanren is often used
as a tool for understanding the guts of a relational language through
studying its implementation. By re-implementing miniKanren as separate
surface syntax over a macro-less and side-effect free
microKanren kernel, we hoped to simultaneously aid implementers when
studying the source code, and also to make the language easier to port
to other hosts that support programming with functions. To support
both of those efforts, we also chose to program in a deliberately
small and workaday set of Scheme primitives.

The sum of those implementation restrictions, however, seemed to force
some awkward choices including: binary logical operators, one at a
time local variable introduction, and leaks in the stream
abstractions. These made the surface syntax macros seem almost
required, and were far enough from our goals that \emph{The Reasoned
  Schemer, 2nd Ed}~\cite{friedman2018reasoned} did not use a
macro-less functional kernel. It also divided host languages into
those that used macros and those that did not. We bridge some of that
split by re-implementing parts of the kernel using some modest runtime
features common to many languages.

Here we:
\begin{itemize}
\item show how to functionally implement more general logical
  operators, cleanly obviating some of the surface macros,

\item survey the design space of macro-less, mostly side-effect free
  implementation alternatives for the remaining macros in the
  \emph{TRS2e} core language implementation, and weigh the trade-offs
  and real-world consequences, and

\item suggest practical solutions for completely eliminating the
  macros in those places where the purest microKanren
  implementations seemed impractical.

\end{itemize}

This resulted in some higher-level (variadic rather than just binary)
operators, a more succinct kernel language, and enabled some
performance improvement. Around half of the changes are applicable to
any microKanren implementation, and the more concise goal combinators
of \cref{sec:conde} may also be of interest to implementers who embed
goal-oriented languages like Icon~\cite{griswold1983icon}. The other
half are necessarily awkward yet practical strategies for those
platforms lacking macro support. The source code for both our
re-implementation and our experimental results is available
at~\url{https://github.com/jasonhemann/tfp-2023/}. Our preferred
definitions are in \cref{mnt:disj-reimplementation} and
\cref{mnt:conda-implementation}.

In \cref{sec:all-aboard}, we illustrate by example what seemed to
force surface syntax macros. In \cref{sec:conde}, we implement
conjunction and disjunction, and in \cref{sec:impure} we discuss the
re-implementation of the impure operators. We discuss the remaining
macros in \cref{sec:functional}. We close with a discussion of the
performance impacts of these implementation choices, and consider how
implementers of Kanren languages in other hosts might benefit from
these alternatives.

\section{The problem}\label{sec:all-aboard}

We briefly reprise here some basics of programming in the miniKanren
implementation of \emph{TRS2e}. Although based on microKanren, the
\emph{TRS2e} implementation makes some concessions to efficiency and
safety and uses a few macros in the language kernel itself. In
addition to that implementation, in this paper we make occasional
references to earlier iterations such as
\citet{hemann2016small}, an expanded
archival version of the 2013 paper~\cite{hemann2013muKanren}.

\subsection{Background}

Programs consist of a database of relation definitions and queries
executed in the context of those definitions. A relation definition
consists of a name for the relation and a number of parameters equal
to the relations arity, and then a body. This body states the
condition under which the relationship holds between the parameters.
Parameters range over a domain of terms, and the programmer can
introduce local auxiliary variables as needed. The core \emph{TRS2e}
language implementation relies on macros \rackinline{fresh},
\rackinline{defrel}, and \rackinline{run} to introduce new logic
variables, globally define relations, and execute queries. Relations
can refer to themselves or one another in their definitions; the whole
database of relations is mutually-recursively defined. The programmer
states the relation body as a logical expression, built up with
conjunctions and disjunctions, over a class of equations and primitive
constraints on terms. Take the \rackinline{carmelit-subway} relation
of \cref{mnt:carmelit} as a concrete example. The Carmelit in Haifa is
the world's shortest subway system with only six stations on its line;
\rackinline{carmelit-subway} is a six-place relation modeling a
passenger riding that subway end to end. \@
\rackinline{carmel-center}, \rackinline{golomb}, \rackinline{masada},
etc.\ are all term constants in the program; each is the name of a
stop on the line. The sub-expression
\rackinline{(== a 'carmel-center)} is an equation between the
parameter \rackinline{a} and the term \rackinline{carmel-center}. The
\rackinline{conde} operator (the \rackinline{e} is for
\enquote{\rackinline{e}very line}) takes any number of parenthesized
sequences of expressions. Each parenthesized sequence represents the
conjunction of those expressions, and the meaning of the
\rackinline{conde} expression represents the disjunction of those
conjunctions. Altogether, this program states that there are two ways
to ride the subway end to end: a passenger can either start at
\rackinline{carmel-center} and travel five stops to
\rackinline{downtown}, or start at \rackinline{downtown} and travel
five stops to \rackinline{carmel-center}.

Each query against the database contains some expression the
programmer wants to satisfy and introduces some number of variables
against which the language should express the answer. Each query asks
either for all, or some bounded number of answers. The
\rackinline{run} expression of \cref{mnt:carmelit} is a query---it
asks for at most three ways a passenger could have ridden the
Carmelit, printed as a list of six values for variables
$\rackinline{s₀} \ldots \rackinline{s₅}$.\@ miniKanren returns a list
of two answers, the only two ways a passenger could ride the whole
subway end to end.

\subsection{Limitations of microKanren}

A six station subway line is an example sufficiently small that
modeling it should be painless. Without the higher-level operators
that the miniKanren macros implement, however, that same relation
requires 11 logical operator nodes, because microKanren only provides
\emph{binary} conjunctions and disjunctions. In our view, this makes
the superficial syntax macros practically mandatory, and impedes host
languages without a macro system. For a logic programming language,
solely binary logical operators is too low level.

\begin{listing}
  \begin{minted}[autogobble,stripall]{racket}
(defrel (carmelit-subway a b c d e f)
  (conde
    ((== a 'carmel-center)
     (== b 'golomb)
     (== c 'masada)
     (== d 'haneviim)
     (== e 'hadar-city-hall)
     (== f 'downtown))
    ((== a 'downtown)
     (== b 'hadar-city-hall)
     (== c 'haneviim)
     (== d 'masada)
     (== e 'golomb)
     (== f 'carmel-center))))

> (run 3 (s₀ s₁ s₂ s₃ s₄ s₅)
    (carmelit-subway s₀ s₁ s₂ s₃ s₄ s₅))
'((carmel-center golomb masada
   haneviim hadar-city-hall downtown)
  (downtown hadar-city-hall haneviim
    masada golomb carmel-center))
  \end{minted}
  \caption{The Carmelit subway relation and a query. Results reformatted for clarity and space.}
  \label{mnt:carmelit}
\end{listing}

Moreover, the microKanren language doesn't offer the programmer
sufficient guidance in using that fine-grained control. With $n$
goals, the programmer can associate to the left, to the right, or some
mixtures of the two. The syntax does not obviously encourage any one
choice. Subtle changes in program structure can have profound effects
on performance, and mistakes are easy to make.

Similarly, the soft-cut operator \rackinline{ifte} in the \emph{TRS2e}
language kernel is also low level. It permits a single test, a single
consequent, and a single alternative. To construct an if-then-else
cascade, a microKanren programmer without the \rackinline{conda}
surface macro would need to code that unrolled conditional expression
by hand.

In earlier renditions of related work, the strictures of pure
functional shallow embedding---without macros---seemed to force
implementations of variable introduction, relation definition, and the
query interface that suffered from harsh downsides. We will revisit
those earlier implementations and their trade-offs, survey the
available options, and suggest compromises for those
truly without macros, thus increasing microKanren's \emph{practical}
portability.

\section{The \textmd{\rackinline{disj}}unction and \textmd{\rackinline{conj}}unction
  goal constructors}\label{sec:conde}

microKanren's binary \rackinline{disj₂} and \rackinline{conj₂}
operators, shown in \cref{mnt:disj2-conj2}, are goal combinators: they
each take two goals, and produce a new goal. A \emph{goal} is what the
program attempts to achieve: it can fail or succeed (and it can
succeed multiple times). A goal executes with respect to a
\emph{state}, here the curried parameter \rackinline{s}, and the
result is a \emph{stream} of states, usually denoted \rackinline{s∞}
as each entry in the stream is a state that results from achieving
that goal in the given state. Disjunction and conjunction work
slightly differently. The \rackinline{append∞} function used in
\rackinline{disj₂} is a kernel primitive that combines two streams
into one, with an interleave mechanism to prevent starvation; the
result is a stream of the ways to achieve the two goals' disjunction.
The \rackinline{append-map∞} function used in \rackinline{conj₂} is to
\rackinline{append∞} what \rackinline{append-map} is to
\rackinline{append}. The ways to achieve the conjunction of two goals
are all the ways to achieve the second goal in a state that results
from achieving the first goal.\@ \rackinline{append-map∞} runs the
second goal over the stream of results from the first goal, and
combines together the results of mapping into a single stream. That
stream represents the conjunction of the two goals, again with special
attention to interleaving and starvation.

\begin{listing}
  \begin{minted}[autogobble,stripall]{racket}
(define ((disj₂ g₁ g₂) s)
  (append∞ (g₁ s) (g₂ s)))

(define ((conj₂ g₁ g₂) s)
  (append-map∞ g₂ (g₁ s)))

(define-syntax disj
  (syntax-rules ()
    ((disj g) g)
    ((disj g₁ g₂ g* ...)
     (disj₂ g₁ (disj g₂ g* ...)))))

(define-syntax conj
  (syntax-rules ()
    ((conj g) g)
    ((conj g₁ g₂ g* ...)
     (conj (conj₂ g₁ g₂) g* ...))))
  \end{minted}
  \caption{microKanren \rackinline{disj₂}, \rackinline{conj₂}, and macros that use them}\label{mnt:disj2-conj2}
\end{listing}

We want to implement disjunction and conjunction as functions taking
arbitrary quantities of goals. These implementations should subsume
the binary \rackinline{disj₂} and \rackinline{conj₂} and they also
should not use \rackinline{apply}. Nor should our implementations
induce a tortured program and extraneous closures to \emph{avoid}
using \rackinline{apply}; we address this somewhat-cryptic requirement
in \cref{sec:whats-left}. This re-implementation requires a host that
supports variable arity functions, a widely available feature included
in such languages as JavaScript, Ruby, Java, and Python. These
languages do not generally support macros, and so we advise
implementers working in such languages to use the present approach.

\subsection{Calculating the solutions}

A developer might derive these definitions as follows. We start with
the definition of a recursive \rackinline{disj} macro like one might
define as surface syntax over the microKanren \rackinline{disj₂}. As
this is not part of the microKanren language, we would like to
dispense with the macro and implement this behavior functionally. At
the cost of an \rackinline{apply}, we can build the corresponding
explicitly recursive \rackinline{disj} function. Since
\rackinline{disj} produces and consumes goals, we can η-expand that
first functional definition by a curried parameter \rackinline{s}. We
then split \rackinline{disj} into two mutually-recursive functions. We
use the symbol \rackinline{⊂} here to indicate that the newer
\rackinline{disj} produces the same values as the old one, although it
now does so by calling a new globally-defined function
\rackinline{D}.\@ This new help function will only ever be called from
\rackinline{disj} so does not \emph{need} to be global; we do so for
space and clarity. In this paper, every such help function we
introduce with \rackinline{⊂} could have instead been local and the
relationship actually an equality.

\begin{minted}[autogobble,stripall,escapeinside=μμ]{racket}
(define (disj g . g*)
  (cond
    ((null? g*) g)
    (else (disj₂ g (apply disj g*)))))
= { μ\text{η-expansion}μ }
(define ((disj g . g*) s)
  (cond
    ((null? g*) (g s))
    (else ((disj₂ g (apply disj g*)) s))))
⊂
(define ((disj g . g*) s)
  (D g g* s))

(define (D g g* s)
  (cond
    ((null? g*) (g s))
    (else ((disj₂ g (apply disj g*)) s))))
\end{minted}

In that version of \rackinline{D}, we can replace the call to
\rackinline{disj₂} by its definition in terms of \rackinline{append∞}
and perform a trivial β-reduction. The explicit \rackinline{s}
argument suggests removing the call to \rackinline{apply} and making
\rackinline{D} self-recursive. The definition of \rackinline{disj}
remains unchanged from before.

\begin{minted}[autogobble,stripall,escapeinside=μμ]{racket}
= { μ\text{by definition of disj₂}μ }
(define (D g g* s)
  (cond
    ((null? g*) (g s))
    (else
     ((λ (s)
        (append∞ (g s) ((apply disj g*) s)))
      s))))
= { μ\text{β-reduction}μ }
(define (D g g* s)
  (cond
    ((null? g*) (g s))
    (else
     (append∞ (g s) ((apply disj g*) s)))))
= { μ\text{by definition of disj}μ }
(define (D g g* s)
  (cond
    ((null? g*) (g s))
    (else
     (append∞ (g s) (D (car g*) (cdr g*) s)))))
\end{minted}

In both clauses of \rackinline{D} we combine \rackinline{g} and
\rackinline{s}, this suggests constructing that stream in
\rackinline{disj} and passing it along. Adding the trivial base case
to that \rackinline{disj} yields the definition
in~\cref{mnt:disj-reimplementation}.

We can derive the definition of \rackinline{conj} from
\cref{mnt:disj-reimplementation} via a similar process. Starting with
the variadic function based on the macro in \cref{mnt:disj2-conj2}, we
first η-expand and split the definition.

\begin{minted}[autogobble,stripall,escapeinside=μμ]{racket}
(define (conj g . g*)
  (cond
    ((null? g*) g)
    (else
     (apply conj
       (cons (conj₂ g (car g*)) (cdr g*))))))
= { μ\text{η-expansion}μ }
(define ((conj g . g*) s)
  (cond
    ((null? g*) (g s))
    (else
     ((apply conj
        (cons (conj₂ g (car g*)) (cdr g*))) s))))
⊂
(define ((conj g . g*) s)
  (C g g* s))

(define (C g g* s)
  (cond
    ((null? g*) (g s))
    (else
     ((apply conj
        (cons (conj₂ g (car g*)) (cdr g*)))
      s))))
\end{minted}

\noindent We next substitute for the definitions of \rackinline{conj}
and \rackinline{conj₂}.

\begin{minted}[autogobble,stripall,escapeinside=μμ]{racket}
(define (C g g* s)
  (cond
    ((null? g*) (g s))
    (else
     ((apply conj
        (cons (conj₂ g (car g*)) (cdr g*)))
      s))))
= { μ\text{by the definition of conj}μ }
(define (C g g* s)
  (cond
    ((null? g*) (g s))
    (else
     (C (conj₂ g (car g*)) (cdr g*) s))))
= { μ\text{by the definition of conj₂}μ }
(define (C g g* s)
  (cond
    ((null? g*) (g s))
    (else
     (C (λ (s)
          (append-map∞ (car g*) (g s)))
        (cdr g*)
        s))))
\end{minted}

This definition of \rackinline{C} sequences the invocations of goal
and state in the order they appear. \rackinline{append-map∞} acts like
a non-deterministic \rackinline{compose} operator. In each recursive
call, we accumulate by mapping, using \rackinline{append-map∞}'s
special delaying implementation of Kanren-language streams, the next
goal in the list.

\begin{minted}[autogobble,stripall,escapeinside=μμ]{racket}
(C g (list g₁ g₂) s)
=
(C (λ (s)
     (append-map∞ g₁
       (g s)))
   (list g₂)
   s)
=
(C (λ (s)
     (append-map∞ g₂
       ((λ (s)
          (append-map∞ g₁
            (g s)))
        s)))
   '()
   s)
=
((λ (s)
   (append-map∞ g₂
     ((λ (s)
        (append-map∞ g₁
          (g s)))
      s)))
 s)
\end{minted}

The state does not change in the recursion: \rackinline{C} only needs
\rackinline{s} to \emph{build} the stream. Therefore we can assemble
the stream on the way in---instead of passing in \rackinline{g} and
\rackinline{s} separately, we pass in their combination as a stream.
The function is tail recursive; we can change the signature in the one
and only external call and the recursive call. Adding the trivial base
case to \rackinline{conj}, yields the version shown in
\cref{mnt:disj-reimplementation}.

\begin{listing}
\begin{minted}[autogobble,stripall,frame=single]{racket}
  (define ((disj . g*) s)
    (cond
      ((null? g*) '())
      (else (D ((car g*) s) (cdr g*) s))))

  (define (D s∞ g* s)
    (cond
      ((null? g*) s∞)
      (else
       (append∞ s∞
         (D ((car g*) s) (cdr g*) s)))))

  (define ((conj . g*) s)
    (cond
      ((null? g*) (cons s '()))
      (else (C (cdr g*) ((car g*) s)))))

  (define (C g* s∞)
    (cond
      ((null? g*) s∞)
      (else
       (C (cdr g*)
          (append-map∞ (car g*) s∞)))))
\end{minted}
  \caption{Final re-definitions of \rackinline{disj} and \rackinline{conj}}\label{mnt:disj-reimplementation}
\end{listing}

Both of these new versions are shallow wrappers over simple folds. The
first steps are to dispense with the trivial case, and then to call a
recursive help function that makes no use of variadic arguments. The
focus is on recurring over the list \rackinline{g*}. Unlike
\rackinline{D}, the function \rackinline{C} does not take in the state
\rackinline{s}; the help function does not need the state for
conjunction.

\subsection{What's left?}\label{sec:whats-left}

More importantly, while both the functional and the macro based
versions of \rackinline{disj} use a right fold, the implementation of
conjunctions in \cref{mnt:disj-reimplementation} uses a left fold over
the goals. This left-fold implementation of conjunctions therefore
left-associates the conjuncts. This is not an accident.

%
Folklore suggests left associating conjunctions tends to improve
performance of miniKanren's interleaving search. The authors know of
no thorough algorithmic proof of such claims, but see for instance
discussions and implementation in
\citet{rosenblatt2019first} for some of
the related work so far. We have generally, however, resorted to small
step visualizations of the search tree to explain the performance
impact. It is worth considering if we can make an equally compelling
argument for this preference through equational reasoning and
comparing the implementations of functions.


% Folklore suggests that left associating conjunctions tends to improve
% the performance of miniKanren's interleaving search. The authors know
% of no thorough algorithmic proof of such claims, but see for instance
% discussions and implementation in~\cite{rosenblatt2019first} for some
% of the related work so far. In \cref{tab:???}, we display the results
% of some micro benchmarks that suggest the same. We have generally,
% however, resorted to small step visualizations of the search tree to
% explain the performance impact. The authors believe it is worth
% considering if we can make an equally compelling argument for this
% preference through equational reasoning and comparing the
% implementations of functions.

Compare the preceding derivation from a left-fold over conjunctions
with the following attempted derivation from a right-fold
implementation. We η-expand and unfold to a recursive help
function like before.

\begin{minted}[autogobble,stripall,escapeinside=μμ]{racket}
(define (conj g . g*)
  (cond
    ((null? g*) g)
    (else (conj₂ g (apply conj g*)))))
= { μ\text{η-expansion}μ }
(define ((conj g . g*) s)
  (cond
    ((null? g*) (g s))
    (else ((conj₂ g (apply conj g*)) s))))
⊂
(define ((conj g . g*) s)
  (C g g* s))

(define (C g g* s)
  (cond
    ((null? g*) (g s))
    (else ((conj₂ g (apply conj g*)) s))))
\end{minted}

In \rackinline{C}, we can substitute in the definition of
\rackinline{conj₂} and β-reduce. We once again η-expand the call to
\rackinline{(apply conj g*)} to substitute with the definition of
\rackinline{conj} and eliminate the \rackinline{apply}. There,
however, we get stuck.

\begin{minted}[autogobble,stripall,escapeinside=μμ]{racket}
(define (C g g* s)
  (cond
    ((null? g*) (g s))
    (else ((conj₂ g (apply conj g*)) s))))
=  { μ\text{by definition of conj₂ and β-reduction}μ }
(define (C g g* s)
  (cond
    ((null? g*) (g s))
    (else
      (append-map∞
        (apply conj g*)
        (g s)))))
= { μ\text{η-expand to use the definition of conj}μ }
(define (C g g* s)
  (cond
    ((null? g*) (g s))
    (else
      (append-map∞
        (λ (s)
          (C (car g*) (cdr g*) s))
        (g s)))))
\end{minted}

At this point in the left-fold derivation, these calls are
accumulating in a stack-like discipline, so we can simplify and pass
\rackinline{g} and \rackinline{s} together as a stream. The equivalent
right-fold implementation, however, is in a kind of
continuation-passing style for non-deterministic computations. Each
goal in the list seems to require a closure for every recursive call.
Building these closures is expensive. Similar behavior shows up in the
right-fold variant of \rackinline{disj}. Basic programming horse sense
suggests the more elegant variants from
\cref{mnt:disj-reimplementation}, and could partly explain the
performance gap.

\begin{minted}[autogobble,stripall,escapeinside=μμ]{racket}
(C g (list g₁ g₂) s)
=
(append-map∞
  (λ (s)
    (C g₁ (list g₂) s))
  (g s))
=
(append-map∞
  (λ (s)
    (append-map∞
      (λ (s)
        (C g₂ '() s))
      (g₁ s)))
  (g s))
=
(append-map∞
  (λ (s)
    (append-map∞
      (λ (s)
        (g₂ s))
      (g₁ s)))
  (g s))
\end{minted}

The new \rackinline{disj} and \rackinline{conj} functions are, we
believe, sufficiently high-level for programmers in implementations
without macros. Though this note mainly concerns working towards an
internal macro-less kernel language, it may also have something to say
about the miniKanren-level surface syntax, namely that even the
miniKanren language could do without its \rackinline{conde} syntax (a
disjunction of conjunctions that looks superficially like
Scheme's \rackinline{cond}) and have the programmer use these new
underlying logical primitives. In \cref{mnt:new-carmelit}, we
implement \rackinline{carmelit-subway} as an example, and it reads
much better than the 11 binary logical operators the programmer would
have needed in the earlier version.

\begin{listing}[h]
  \begin{minted}[autogobble,stripall]{racket}
(defrel (carmelit-subway a b c d e f)
  (disj
    (conj (== a 'carmel-center)
          (== b 'golomb)
          (== c 'masada)
          (== d 'haneviim)
          (== e 'hadar-city-hall)
          (== f 'downtown))
    (conj (== a 'downtown)
          (== b 'hadar-city-hall)
          (== c 'haneviim)
          (== d 'masada)
          (== e 'golomb)
          (== f 'carmel-center))))
  \end{minted}
  \caption{A new Carmelit subway without \rackinline{conde}}\label{mnt:new-carmelit}
\end{listing}

\section{Tidying up the impure operators}\label{sec:impure}

The \rackinline{conda} of \emph{TRS2e} provides nested
\enquote{if-then-else} behavior (the \rackinline{a} is because
\rackinline{a}t most one line succeeds). It relies on microKanren's
underlying \rackinline{ifte}. That \rackinline{conda} requires one or
more conjuncts per clause and one or more clauses. Once again, we
would like to have an equivalently expressive feature without
resorting to macros. Unlike the \emph{TRS2e} implementation, the
versions of \rackinline{conda} in \cref{mnt:conda-macro} consume a
sequence of goals. They consume those goals in \enquote{if-then}
pairs, perhaps followed by a final \enquote{else}; we have no choice
if we want to proceed without using a macro.

\begin{listing}
\begin{minted}[autogobble,stripall]{racket}
(define-syntax conda
  (syntax-rules ()
    ((conda g) g)
    ((conda g₁ g₂) (conj g₁ g₂))
    ((conda g₁ g₂ g₃ g* ...)
     (ifte g₁ g₂ (conda g₃ g* ...)))))

(define (conda g . g*)
  (cond
    ((null? g*) g)
    ((null? (cdr g*)) (conj g (car g*)))
    (else
     (ifte g (car g*) (apply conda (cdr g*))))))
\end{minted}
  \caption{A re-implemented \rackinline{conda} macro and its functional equivalent}\label{mnt:conda-macro}
\end{listing}

We defer the derivation of our new \rackinline{conda} solution to
\cref{sec:conda-derivation}. Rather than building a largely redundant
implementation of \rackinline{condu}, we expose the higher-order goal
\rackinline{once} to the user. The programmer can simulate
\rackinline{condu} by wrapping \rackinline{once} around every test
goal. We do however take this opportunity to also lift the inner
function \rackinline{loop} to a top-level definition.

\begin{listing}[h]
  \begin{minted}[autogobble,stripall,frame=single]{racket}
(define ((conda . g*) s)
  (cond
    ((null? g*) '())
    (else (A (cdr g*) ((car g*) s) s))))

(define (A g* s∞ s)
  (cond
    ((null? g*) s∞)
    ((null? (cdr g*)) (append-map∞ (car g*) s∞))
    (else (ifs∞te s∞ (car g*) (cdr g*) s))))

(define (ifs∞te s∞ g g+ s)
  (cond
    ((null? s∞) (A (cdr g+) ((car g+) s) s))
    ((pair? s∞) (append-map∞ g s∞))
    (else (λ () (ifs∞te (s∞) g g+ s)))))

(define ((once g) s)
  (O (g s)))

(define (O s∞)
  (cond
    ((null? s∞) '())
    ((pair? s∞) (cons (car s∞) '()))
    (else (λ () (O (s∞))))))
  \end{minted}
  \caption{A functional \rackinline{conda}, \rackinline{ifte}, and \rackinline{once}}\label{mnt:conda-implementation}
\end{listing}

\section{Removing more macros}\label{sec:functional}

The \citeyear{hemann2013muKanren} microKanren paper demonstrates how
to implement a side-effect free macro-less Kanren language in an eager
host. In \cref{mnt:call-fresh-and-call-initial-state} we display these
alternative mechanisms for introducing fresh logic variables,
executing queries, and introducing delay and interleave. The versions
in \cref{mnt:call-fresh-and-call-initial-state} are slightly adjusted
to be consistent with this presentation.

\begin{listing}
  \begin{minted}[autogobble,stripall]{racket}
(define ((call/fresh f) s)
  (let ((v (state->newvar s)))
    ((f v) (incr-var-ct s))))

(define (call/initial-state n g)
  (reify/1st
    (take∞ n (pull (g initial-state)))))

(define (((Zzz g) s))
  (g s))

> (call/initial-state 3
    (call/fresh
      (λ (x)
        (== x 'cat))))
  \end{minted}
  \caption{Definition and use of functional microKanren equivalents of \emph{TRS2e} kernel macros}\label{mnt:call-fresh-and-call-initial-state}
\end{listing}

Each of these has drawbacks that compelled the \emph{TRS2e} authors to
instead use macro-based alternatives in the kernel layer. In this
section, we explicitly address those drawbacks and point out some
other non-macro alternatives that may demand more from a host language
than the original microKanren choices, and make some recommendations.

\subsection{Logic variables}

Many of the choices for these last options hinge on a representation
of logic variables. Every implementation must have a mechanism to
produce the next fresh logic variable. The choice of variable
representation will affect the implementation of unification and
constraint solving, the actual introduction of fresh variables, as
well as answer projection, the formatting and presentation of a
query's results. Depending on the implementation, the variables may
also need additional functions to support them. In a shallow
embedding, designing a set of logic variables means either using a
\rackinline{struct}-like mechanism to custom-build a datatype hidden
from the microKanren programmer, or designating some subset of the
host language's values for use as logic variables. Using
\rackinline{struct}s and limiting the visibility of the constructors
and accessors is a nice option for languages that support it.

The choice of which host language values to take for logic variables
divides roughly into the structurally equal and the referentially
equal. For an example of the latter, consider representing each
variable using a vector, and identifying vectors by their unique
memory location. This latter approach models logic variables as a
single global pool rather than reused separately across each disjunct,
and so requires more logic variables overall. The microKanren approach
uses natural numbers as an indexed set of variables, which
necessitates removing numbers from the user's term language.

\subsection{\rackinline{fresh}}

There are numerous ways to represent variables, and so too are there
many ways to introduce fresh variables. In the microKanren approach,
the current variable index is one of the fields of the state threaded
through the computation; to go from index to variable is the identity
function, and the \rackinline{state->newvar} function we use can be
just an accessor. The function \rackinline{incr-var-ct} can
reconstruct that state with the variable count incremented. The
\rackinline{call/fresh} function, shown in
\cref{mnt:call-fresh-and-call-initial-state}, takes as its first
argument a goal parameterized by the name of a fresh variable.
\rackinline{call/fresh} then applies that function with the newly
created logic variable, thereby associating that host-language lexical
variable with the DSL's logic variable. This lets the logic language
\enquote{piggyback} on the host's lexical scoping and variable lookup,
as shown in \cref{mnt:call-fresh-and-call-initial-state}.

This approach also means, however, that absent some additional
machinery the user must introduce those new logic variables one at a
time, once each per \rackinline{call/fresh} expression, as though
manually currying all functions in a functional language. This made
programs larger than the relational \rackinline{append} difficult to
write and to read, and that amount of threading and re-threading state
for each variable is costly. We could easily support, say instead,
three variables at a time---force the user to provide a three-argument
function and always supply three fresh variables at a time. Though
practically workable the choice of some arbitrary quantity $k$ of
variables at a time, or choices $k_{1}$ and $k_{2}$ for that matter,
seems unsatisfactory. It could make sense to inspect a procedure for
its arity at runtime and introduce exactly that many variables, in
languages that support that ability. In many languages, a procedure's
arity is more complex than a single number. Variadic functions and
keyword arguments all complicate the story of a procedure's arity. A
form like \rackinline{case-lambda} means that a single procedure may
have several disjoint arities. The arity inspection approach could be
a partial solution where the implementer restricts the programmer to
using functions with fixed arity.

One last approach is to directly expose to the user a mechanism to
create a new variable, and allow the programmer to use something like
a \rackinline{let} binding to do their own variable introduction and
name binding. Under any referentially transparent representation of
variables, this would mean that the programmer would be responsible
for tracking the next lexical variable. This last approach pairs best
with referentially opaque variables where the operation to produce a
new variable allocates some formerly unused memory location so the
programmer does not need to track the next logic variable. See
sokuza-kanren~\cite{kiselyov2006taste} for an example of this style.
With this latter approach, however, we can expose \rackinline{var}
directly to the programmer who can use \rackinline{let} bindings to
introduce several logic variables simultaneously.

\subsection{\rackinline{run}}

\Cref{mnt:call-fresh-and-call-initial-state} also shows how we have
implemented a \rackinline{run}-like behavior without using macros.
Using a referentially transparent implementation of logic variables,
we can accomplish the job of \rackinline{run} and \rackinline{run*} by
a \rackinline{call/initial-state}-like function. The query is itself
expressed as a goal that introduces the first logic variable
\rackinline{q}. A \rackinline{run}-like operator displays the result
with respect to the first variable introduced. This means pruning
superfluous variables from the answer, producing a single value from
the accumulated equations, and numbering the fresh variables. When
logic variables are only identified by reference equality, the
language implementation must pass the same \emph{pointer-identical}
logic variable into both the query and into the answer projection,
called \rackinline{reify}. The pointer-based logic variable approach
forces the programmer to explicitly invoke \rackinline{reify} as
though it were a goal as the last step of executing the query, as in
the first example in \cref{mnt:run-query}, or create a special
variable introduction mechanism for the first variable, scoped over
both the query and the answer projection, as in the second example.

\begin{listing}[h]
  \begin{minted}[autogobble,stripall]{racket}
(call/initial-state 1
  (let ((q (var 'q)))
    (conj
      (let ((x (var 'x)))
        (== q x))
      (reify q))))

(define (call/initial-state n f)
  (let ((q (var 'q)))
    (map (reify q)
         (take∞ n ((f q) initial-state)))))
  \end{minted}
  \caption{Several approaches to reifying variables in \rackinline{call/initial-state}. Here \rackinline{initial-state} is a representation of an initially empty set of equations}\label{mnt:run-query}
\end{listing}



\subsection{\rackinline{define}}

The microKanren programmer can use their host language's
\rackinline{define} feature to construct relations as host-language
functions, and manually introduce the delays in relations using a help
function like \rackinline{Zzz}
(\cref{mnt:call-fresh-and-call-initial-state}) to introduce delays, as
in the original implementation.~\cite{hemann2013muKanren} This may be
a larger concession than it looks, since it exposes the delay and
interleave mechanism to the user, and both correct interleaving and,
in an eager host language, even the termination of relation
\emph{definitions} rely on a whole-program correctness property of
relation definitions having a delay. \rackinline{Zzz} \emph{if always
  used correctly} would be sufficient to address that problem, but
forgetting it just once could cause the entire program to loop.
Turning the delaying and interleaving into a user-level operation
means giving the programmer some explicit control over the search, and
that in turn could transform a logic language into an imperative one.
Another downside of relying on a host-language \rackinline{define} is
that the programmer must now take extra care not to provide multiple
goals in the body. The \rackinline{define} form will treat all but the
last expression as statements and silently drop them, rather than
conjoin them as in \rackinline{defrel}. For those implementing a
shallowly embedded stream-based implementation in an eager host
language, that can be a subtle mistake to debug.

\section{Future work}\label{sec:conclusion}

This note shows how to provide a somewhat more concise core language
that significantly reduces the need for macros, and provides some
alternatives for those working without macros that may be more practical
than those of~\citet{hemann2013muKanren}.

Forcing ourselves to define \rackinline{disj} and \rackinline{conj}
functionally, and with the restrictions we placed on ourselves in this
re-implementation, removed a degree of implementation freedom and led
us to what seems like the right solution. The result is closer to the
design of Prolog, where the user represents conjunction of goals in
the body of a clause with a comma and disjunction, either implicitly
in listing various clauses or explicitly with a semicolon. The prior
desugaring macros do not seem to suggest how to associate the calls to
the binary primitives---both left and right look equally nice---where
these transformations suggest a reason for the performance difference.
The functional \rackinline{conda} re-implementation is now also
variadic, and exposing \rackinline{once} to the programmer makes using
committed choice almost as easy as with the
earlier \rackinline{condu}.

Existing techniques for
implementing \rackinline{defrel}, \rackinline{fresh},
and \rackinline{run} (and \rackinline{run*}) without macros have
serious drawbacks. They include exposing the implementation of
streams and delays, and the inefficiency and clumsiness of introducing
variables one at a time, or the need to reason about global state. With
a few more runtime features from the host language, an implementer can
overcome some of those drawbacks, and may find one of the suggested
proposals an acceptable trade-off.

From time to time we find that the usual miniKanren implementation is
\emph{itself} lower-level than we would like to program with
relations. Early microKanren implementations restrict themselves
to \rackinline{syntax-rules} macros. Some programmers use macros to
extend the language further as
with \rackinline{matche}~\cite{keep2009pattern}. Some constructions
over miniKanren, such
as \rackinline{minikanren-ee}~\cite{ballantyne2020macros}, may rely on
more expressive macro systems
like \rackinline{syntax-parse}~\cite{culpepper2012fortifying}.

We would still like to know if our desiderata here are \emph{causally}
related to good miniKanren performance. Can we reason at the
implementation level and peer through to the implications for
performance? If left associating \rackinline{conj} is indeed uniformly
a dramatic improvement, the community might consider reclassifying
left-associative conjunction as a matter of correctness rather than an
optimization, as in \enquote{tail call optimization} vs.
\enquote{Properly Implemented Tail Call
  Handling}~\cite{felleisen2014requestions}. Regardless, we hope this
document narrows the gap between macro-free microKanrens and those
using macro systems, and leads to more elegant, expressive and
efficient implementations regardless of functional host language.

\begin{acks}

  Thanks to Michael Ballantyne, Greg Rosenblatt, Ken Shan, and Jeremy
  Siek for their helpful discussions and ideas. Our thanks to Yafei
  Yang and Darshal Shetty for their implementation suggestions. We
  would also like to thank our anonymous reviewers for their
  insightful contributions.

\end{acks}

\printbibliography{}

\appendix

\section{\rackinline{conda} derivation}\label{sec:conda-derivation}

\begin{minted}[autogobble,stripall,escapeinside=μμ]{racket}
(define (conda g . g*)
  (cond
    ((null? g*) g)
    ((null? (cdr g*)) (conj g (car g*)))
    (else
     (ifte g (car g*) (apply conda (cdr g*))))))
= { μ\text{η-expansion}μ }
(define ((conda g . g*) s)
  (cond
    ((null? g*) (g s))
    ((null? (cdr g*)) ((conj g (car g*)) s))
    (else
     ((ifte g (car g*) (apply conda (cdr g*)))
      s))))
⊂
(define ((conda g . g*) s)
  (A g g* s))
(define (A g g* s)
  (cond
    ((null? g*) (g s))
    ((null? (cdr g*)) ((conj g (car g*)) s))
    (else
     ((ifte g (car g*) (apply conda (cdr g*)))
      s))))
= { μ\text{let bindings and η-expansion}μ }
(define (A g g* s)
  (cond
    ((null? g*) (g s))
    (else
     (let ((g1 (car g*)) (g* (cdr g*)))
       (cond
         ((null? g*) ((conj g g1) s))
         (else
          ((ifte g g1
             (λ (s)
               ((apply conda g*) s)))
           s)))))))
= { μ\text{by definition of conda, conj, and C}μ }
(define (A g g* s)
  (cond
    ((null? g*) (g s))
    (else
     (let ((g1 (car g*)) (g* (cdr g*)))
       (cond
         ((null? g*) (append-map∞ g1 (g s)))
         (else
          ((ifte g g1
             (λ (s)
               (A (car g*) (cdr g*) s)))
           s)))))))
= { μ\text{by definition of ifte and a β reduction}μ }
(define (A g g* s)
  (cond
    ((null? g*) (g s))
    (else
     (let ((g1 (car g*)) (g* (cdr g*)))
       (cond
         ((null? g*) (append-map∞ g1 (g s)))
         (else
          (let loop ((s∞ (g s)))
            (cond
              ((null? s∞) (A (car g*) (cdr g*) s))
              ((pair? s∞) (append-map∞ g1 s∞))
              (else (λ () (loop (s∞))))))))))))
\end{minted}

At this point \rackinline{g} and \rackinline{s} in \rackinline{conda}
are \emph{begging} to be passed as a stream \rackinline{s∞}; we oblige
them. We lift that local function \rackinline{loop} to a global
definition, passing all the parameters it needs. Since the only call
to \rackinline{ifs∞te} is in \rackinline{A}, we know that
\rackinline{ifs∞te}'s third parameter will always be a non-empty list.

\begin{minted}[autogobble,stripall,escapeinside=μμ]{racket}
(define (A g* s∞)
  (cond
    ((null? g*) s∞)
    (else
     (let ((g1 (car g*)) (g* (cdr g*)))
       (cond
         ((null? g*) (append-map∞ g1 s∞))
         (else
          (let loop ((s∞ s∞))
            (cond
              ((null? s∞) (A (cdr g*) ((car g*) s)))
              ((pair? s∞) (append-map∞ g1 s∞))
              (else (λ () (loop (s∞))))))))))))
⊂
(define (A g* s∞)
  (cond
    ((null? g*) s∞)
    (else
     (let ((g1 (car g*)) (g* (cdr g*)))
       (cond
         ((null? g*) (append-map∞ g1 s∞))
         (else (ifs∞te s∞ g1 g* s)))))))

(define (ifs∞te s∞ g g+ s)
  (cond
    ((null? s∞) (A (cdr g+) ((car g+) s)))
    ((pair? s∞) (append-map∞ g s∞))
    (else (λ () (ifs∞te (s∞) g g+ s)))))
\end{minted}

If we also add a line in \rackinline{conda} to dispatch with the
trivial case, we arrive at the definition in
\cref{mnt:conda-implementation}. Most of
\cref{mnt:conda-implementation} is a functional implementation of that
cascade behavior. \rackinline{A} knows it has at least one goal; it's
job is to determine if there is precisely one goal, precisely two
goals, or more than two goals.


\end{document}


%%% Local Variables:
%%% mode: latex
%%% TeX-master: t
%%% End:
